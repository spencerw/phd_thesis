\chapter {Introduction}

\section{Pathways to Terrestrial Planet Formation}

The growth and eventual formation of terrestrial planets, which are classified by their composition of primarily rock and metal, is a process that spans a huge range of sizes in a bottom-up fashion \cite{safronov72}. This process beings with $\mu$m sized dust grains and eventually leads to gravitationally-bound bodies which are 1000's of km in size. Given the wide range of size scales that objects evolve through as they grow to form planets, there is a diverse collection of physical processes that must be considered and modeled to properly capture the possible outcomes of the terrestrial planet formation process.

At the smallest sizes, aerodynamic forces between solids and the gaseous component of the protoplanetary disk dominate growth. Material physics is important here, as grains sediment toward the disk midplane and occasionally collide with each other and stick \cite{okuzumi12, windmark12, garaud13, katoka13} to progressively form larger bodies.  Around mm size scales, however, a number of growth barriers are thought to present themselves. These include catastrophically short radial drift timescales \cite{adachi76, weidenschilling77}, problems with particles sticking to form larger objects and even destructive collisions \cite{windmark12}.

Despite these barriers to growth, terrestrial planet formation appears common \cite{bonfils13, dressing15, gaidos16}. This implies that some physical process must quickly and efficiently grow solids up to kilometer sizes, where gravity begins to dominate the physical interactions. These small gravitationally bound objects are referred to as \textit{planetesimals} and are usually taken to be the basic building blocks of terrestrial planets. In planet formation models, planetesimals are allowed to collide and grow, while gravitational interactions and aerodynamic drag from the residual gaseous disk alter their velocities. Although most planetesimals are eventually incorporated into planets, a small amount can remain long after the system forms. In the solar system, these bodies eventually went on to populate the asteroid belt and Kuiper belt \cite{duncan89, bottke05, levison08, morbidelli09}. In some adolescent exoplanetary systems, dust from these colliding residual planetesimals is often observable \cite{wyatt08, gaspar20}.

A significant uncertainty in the final planet formation process involves determining whether and how far these building blocks move throughout the disk as they incorporate themselves into the final bodies. Although radial drift due to aerodynamic gas drag is can be treated as a second-order effect for planetesimal and larger-sized objects, objects that reach roughly earth-sized can stir up spiral density waves in the gas disk and lose significant amounts of angular momentum as they migrate inward \cite{ward97}. This has significant implications for the final orbital architecture of the planetary system, as the solid mass becomes much more centrally concentrated and planetary compositions consist of material that should be expected to condense much further from the central star. One key prediction of a migration-driven model is that planets should be expected to stop migrating and lock into mean-motion resonances once they reach the inner edge of the gaseous disk \cite{hands14}. An appreciable fraction of close-in terrestrial multiplanet systems are found in resonant chains \cite{gillon16, gillon17, christiansen18, agol21, leleu21}. However, resonant chain systems formed in this fashion are often not dynamically stable and this telltale signature of convergent migration is only temporary \cite{terquem07, pierens11, izidoro17, mcnally19}.

Alternatively, (talk about in situ formation)

% Note for later: a lot of the open questions (planetesimal formation, type i migration) depend on the structure of the gas and solids in the disk, which we have no observational constraints on

\section{Planetesimal Formation Processes}
\section{Observational Constraints}
\section{Modeling Planetesimal Growth}
\section{Thesis Goals and Outline}