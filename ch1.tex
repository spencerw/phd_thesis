\chapter {Introduction}

\section{Pathways to Terrestrial Planet Formation}

The growth and eventual formation of terrestrial planets, which are classified by their composition of primarily rock and metal, is a process that spans a huge range of sizes in a bottom-up fashion \cite{safronov72}. This process beings with $\mu$m sized dust grains and eventually leads to gravitationally-bound bodies which are 1000's of km in size. Given the wide range of size scales that objects evolve through as they grow to form planets, there is a diverse collection of physical processes that must be considered and modeled to properly capture the possible outcomes of the terrestrial planet formation process.

At the smallest sizes, aerodynamic forces between solids and the gaseous component of the protoplanetary disk dominate growth. Material physics is important here, as grains sediment toward the disk midplane and occasionally collide with each other and stick \cite{okuzumi12, windmark12, garaud13, katoka13} to progressively form larger bodies.  Around mm size scales, however, a number of growth barriers are thought to present themselves. These include catastrophically short radial drift timescales \cite{adachi76, weidenschilling77}, problems with particles sticking to form larger objects and even destructive collisions \cite{windmark12}.

Despite these barriers to growth, terrestrial planet formation appears common \cite{bonfils13, dressing15, gaidos16}. This implies that some physical process must quickly and efficiently grow solids up to kilometer sizes, where gravity begins to dominate the physical interactions. These small gravitationally bound objects are referred to as \textit{planetesimals} and are usually taken to be the basic building blocks of terrestrial planets. In planet formation models, planetesimals are allowed to collide and grow, while gravitational interactions and aerodynamic drag from the residual gaseous disk alter their velocities. Although most planetesimals are eventually incorporated into planets, a small amount can remain long after the system forms. In the solar system, these bodies eventually went on to populate the asteroid belt and Kuiper belt \cite{duncan89, bottke05, levison08, morbidelli09}. In some adolescent exoplanetary systems, dust from these colliding residual planetesimals is often observable \cite{wyatt08, gaspar20}.

A significant uncertainty in the final planet formation process involves determining whether and how far these building blocks move throughout the disk as they incorporate themselves into the final bodies. Although radial drift due to aerodynamic gas drag is can be treated as a second-order effect for planetesimal and larger-sized objects, objects that reach roughly Mars-sized can stir up spiral density waves in the gas disk and lose significant amounts of angular momentum as they migrate inward \cite{ward97}. This has significant implications for the final orbital architecture of the planetary system, as the solid mass becomes much more centrally concentrated and planetary compositions consist of material that should be expected to condense much further from the central star. One key prediction of a migration-driven model is that planets should be expected to stop migrating and lock into mean-motion resonances once they reach the inner edge of the gaseous disk \cite{hands14}. An appreciable fraction of close-in terrestrial multiplanet systems are found in resonant chains \cite{gillon16, gillon17, christiansen18, agol21, leleu21}. However, resonant chain systems formed in this fashion are often not dynamically stable and this telltale signature of convergent migration is often only temporary \cite{terquem07, pierens11, izidoro17, mcnally19}.

Alternatively, large-scale migration due to tidal torques with the gas disk may play an insignificant role in shaping the final planetary configuration and terrestrial planets are formed largely \textit{in-situ}. In this scenario, the final properties of the planets themselves more closely reflect the conditions of the initial disk of solids (at least at the time of planetesimal formation). A common method to assess the viability of in-situ formation is to take an observed collection of planets, smear out the masses, and then evaluate the physicality of the inferred distribution of solids. This approach was first applied to the solar system to infer the natal distribution of solids (called the MMSN, or minimum-mass solar nebula) \cite{hayashi81} and has more recently been applied to compact multiplanet systems observed by Kepler \cite{chiang13, dai20}. One attractive feature of an in-situ model is it's relative simplicity to implement in planet formation models. Gravity and collisions are the two physical processes that dominate, both of which can easily be handled by modern N-body codes. Gas disk migration models, on the other hand, are still wrought with uncertainties that bring into question the strength, timing and even direction of migration that depend on both the exact mass distribution and thermodynamic properties of the gas disk \cite{ayliffe10, bitsch13, ogihara18}.

The wealth of information that recent missions like Kepler and TESS have provided about terrestrial planets is making one fact abundantly clear: a one-size-fits-all model for terrestrial planet formation is overly simplistic. In some cases, it it clear that in-situ formation cannot have operated \cite{raymond14}, while only a small fraction of systems have clearly undergone large-scale migration \cite{he22}. Given the limitations and uncertainties associated with current migration models, and the relative simplicity of growing gravitationally bound solids in-situ, a closer look at planetesimal accretion and eventual terrestrial planet formation using high resolution N-body simulations is warranted. Until recently, these types of simulations have required overly restrictive simplifications, such as starting with fully-formed Mars-sized planetary embryos, or ignoring gravitational interactions between smaller bodies. In this thesis, I present some of the first simulations that are free of these limitations and use the results to understand small body populations in the present day solar system, collisionally generated dust around young planetary systems and compact multi planet systems observed by Kepler and TESS (also known as STIPs, or systems of tightly-packed inner planets).

% Note for later: a lot of the open questions (planetesimal formation, type i migration) depend on the structure of the gas and solids in the disk, which we have no observational constraints on

\section{Planetesimal Formation}

As mentioned above, the barriers to growth for mm-sized bodies due to radial drift, bouncing and fragmentation suggest that continued growth through cm and meter sizes must proceed in a qualitatively different fashion. For mm-sized objects, equal mass collisions do not tend to result in constructive growth \cite{windmark12}. Even if this barrier is circumvented, cm and meter sized bodies drift through the gas disk and fall onto the central star on timescales much shorter than the typical collision timescale \cite{weidenschilling77}. Given the fact that solids do manage to eventually grow beyond mm sizes in planet forming disks, the growth from mm to $\sim$ km sizes must be fast and efficient.

The most straightforward way to induce growth to overcome the bouncing, fragmentation and drift barriers is to concentrate solids to densities above the Roche density and induce a gravitational instability. One obvious way to do this is to wait for solid particles to sediment to the midplane of the disk \cite{goldreich73}. Unfortunately, gas turbulence at the disk midplane prevents solid densities from ever reaching sufficiently high concentrations \cite{cuzzi93}. 

 A solution to this problem that has gained a huge amount of traction is known as the streaming instability \cite{youdin05}. With the streaming instability, the solids induce a backreaction on the gas orbiting at sub-Keplerian speeds, which in turn slows down radial drift of the solids and causes them to pile up. This, of course, causes an ever stronger backreaction on the gas and the cycle continues as the local concentration of solids grows. Numerical studies have shown that this process can produce particle concentrations sufficient for gravitational collapse and typically generates planetesimals on the order of 100 km in size \cite{johansen15, simon16, schafer17}. Interestingly, the planetesimals often form in binaries \cite{li19} and bear a striking resemblance to present day Kuiper belt objects.
 
 % Add figure from Li paper showing collapse of SI filaments
 
Unfortunately, the conditions in simple models of protoplanetary disks are not sufficient to trigger the streaming instability. High resolution numerical studies suggest that dust to gas ratios larger than 4 percent are required to trigger it \cite{carrera15, yang17}, while the early solar system was thought to have a dust to gas ratio closer to 1 percent \cite{hayashi81}. This suggests that planetesimals should not be expected to form everywhere in the disk, although the fact that planets and smaller bodies, which presumably originated from planetesimals, exists everywhere from $\sim$ 0.5 to $\sim$ 50 AU in the solar disk. Future, higher resolution studies of the streaming instability may alleviate this discrepancy.

Another solution to the problem of triggering the streaming instability may come from relaxing the assumption that the terrestrial planet forming regions of protoplanetary disks are smoothly varying. Pressure bumps in the gas disk, formed by mechanisms such as ionization fronts, condensation fronts, perturbations from other planets or self-induced dust traps \cite{gonzalez17} can all act to halt the inward drift of small solids and concentrate them into rings with conditions amenable to triggering the streaming instability of even gravitational fragmentation.

\section{Observational Constraints}

For $\mu$m to mm sized grains, scattered light from the central star, along with thermal emission allows us to directly trace this early phase of growth. For Earth-sized planets and larger, alterations to the light emitted by the host star allow us to observe the final outcome of the planet formation process. Populations of objects in between these sizes, however, do not have enough surface area to scatter light or emit appreciable amounts of thermal radiation, while being too small to perturb the light from the central star. With the exception of the remaining small bodies in the solar system, many of which are believed to be residual planetesimals, the gravitationally-dominated stage of the terrestrial planet formation process is essentially invisible.

In the solar system, the orbital configurations of the planets themselves contain clues that the small body population was once much more numerous. A model in which the orbits of Saturn and Jupiter gradually evolve due to repeated scattering events with smaller bodies and subsequently go unstable works well to explain the present day configuration of the Jupiter trojans, Kuiper belt objects and irregular satellites (cite Nice model). This suggests that the solar system was once filled with planetesimals. In addition, N-body simulations have shown that a collection 100 km sized objects placed throughout the outer solar system can induce the inward migration of Neptune's orbit while capturing the proper amount of small bodies into mean-motion resonance \cite{murrayclay06}.

For planet-forming disks around other stars, the primary way to infer the presence of planetesimals and planetary embryos is through dust emission. Although some of this dust is likely primordial, a large fraction of it is expected to be generated by collisions between the gravitationally bound planetary building blocks. Although the initial stages of planetesimal growth are not expected to create much dust, the eventual protoplanets that form act to stir up the remaining planetesimals, increasing collision velocities and producing destructive collisions \cite{kenyon04}. In 2020, the thermal emission of a transient dust cloud resulting from a planetesimal-planetesimal collision was observed for the first time \cite{gaspar20}.

% Fomalhut B image

Often, the collisionally generated dust emission is used to infer the presence of an unseen planet. The intricate structure of bright rings and gaps seen in the planet-forming disk HL Tau \cite{alma15} have been argued to be indicative of three embedded giant planets \cite{boley17}. \cite{dobinson13, dobinson16} showed that the presence of a giant planet produces distinct morphological features in the collisionally generated dust emission, the properties of which can be used to constrain the orbital properties of the planet itself.

% Final point: no observational contrains on the terrestrial region of planet forming disks

\section{Modeling Planetesimal Growth}
\section{Thesis Goals and Outline}