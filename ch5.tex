\chapter {In-Situ Formation of STIPs}

%Intro outline:
% What are STIPs, why important
% Outline of formation theories, migration vs in-situ
% Problems with migration models, in-situ simpler
% Outline my previous paper
% Motivation: does accretion boundary leave any imprint on final orbital arcitectures?

\section{Introduction} \label{sec:intro}

One of the most surprising and intriguging results from NASA's \textit{Kepler} mission was the prevalance 
of sub-Neptune-sized exoplanets with orbital periods shorter than that of Mercury \cite{borucki10}. These 
planets are often found in closely-spaced groups, known as systems of tightly-packed inner planets (
STIPs) \cite{latham11, lissauer11, lissauer14}. STIPs are found to often be coplanar 
\cite{fang12, tremaine12}, have low eccentricities \cite{vaneylen15, hadden17} and are usually not in 
mean-motion resonance with each other \cite{fabrycky14, steffen15}. In addition these systems have been 
shown to exhibit a compelling amount of uniformity between adjacent planets, in terms of mass, radius and 
orbital spacing \cite{millholland17, millholland21}.

All of these characteristics provide a rich set of constraints for planet formation models. One of the 
most intriguing questions that arises is why the structure of the present-day solar system appears so 
different than that of STIPs. Extending the minimum-mass solar nebular (MMSN) model \cite{hayashi81} down 
to $\sim$ 0.05 AU, where the inner edge of many protoplanetary disks are thought to be \cite{meyer97} (
and where planetesimals are thought to form \cite{mulders18}) yields several extra Earth masses of 
material. Explaining why this material is missing from the solar system, but not some exoplanetary 
systems, is a key question that a complete theory of planet formation must be able to provide.

There are two categories of formation theories meant to explain the ubiquity of these compact, short-period systems. The first 
involves the gradual assembly of these worlds as they migrate inwards. For Mars-sized bodies and larger, torques at the Lindblad 
resonances drive the exchange of angular momentum between the gaseous disk and the planet \cite{ward97}. A key prediction of 
migration models is that multiplanet systems should end up in resonant chains \cite{cresswell06}. Although this is not unheard of 
(see Kepler 223 \cite{mills16} and Trappist-1 \cite{gillon16, gillon17, agol21}), the majority of these systems are found to not be in 
resonant chains \cite{lissauer11, fabrycky14}. One potential explanation for this discrepancy involves a later phase of 
destabilization after convergent migration has completed and the gaseous disk subsequently dissipates 
\cite{izidoro17, izidoro21}. However, the detailed behavior of tidal torque-driven migration is still very uncertain and it is not 
entirely clear when and how convergent migration should operate. The strength and even direction of tidal migration depends 
sensitively on the thermodynamic structure of the gas disk \cite{ayliffe10, bitsch13}.

Alternatively, these planets may have formed in situ, having been constructed only from the local mass budget of the disk. This appears to be the case for the gaseous envelopes of hot Jupiters formed near the inner edge of the disk \cite{bailey18}, although it is not clear whether the cores of these worlds require a migration model. Currently, there are no measurements constraining the mass budget of the innermost regions of planet-forming disks, so the initial conditions for in situ models tend to rely on wild speculation. By enhancing the solid surface density by a factor of $\sim$ 100 relative to the inner solar system, \cite{hansen12} was able to form compact multiplanet super-Earth systems without invoking any kind of planet migration. In situ formation may be particularly prevalent around low mass stars, as magnetically-driven disk winds tend to flatten out the radial gas density profile, which balances out the inner and outer torques from tidally-driven migration \cite{ogihara18}. One should note that no in situ models have yet to produce resonant chains of planets, so it seems likely that gas disk migration likely plays a role at least for the subset of planetary systems found in this type of configuration.

% Text from NSF proposal:

%With the number of confirmed exoplanets now exceeding 4000, there is now a large enough statistical sample from which to test planet formation theories. The most striking trend in this sample is the prevalence of planets with orbital periods shorter than that of Mercury. Many of these planets come in closely-packed multiples \citep{2014ApJ...790..146F}, known as systems of tightly packed inner planets (STIPs). So far, nearly all STIPs have been found around M stars \citep{2015ApJ...807...45D}. Extending the minimum-mass solar nebula (MMSN) model \citep{1981PThPS..70...35H} down to $\approx$ 0.05 AU, where the inner edge of many protoplanetary disks are thought to be \citep{1997AJ....114..288M} yields several extra Earth masses of material. Explaining why this material is missing from the solar system, but not some exoplanetary systems, is a key question that a complete theory of planet formation must be able to explain.

%There are two categories of theories meant to explain the ubiquity of these compact, short period systems. The first involves the gradual assembly of these worlds as they migrate inwards. Bodies larger than Mars are expected to experience strong torques from the gas disk that drive a substantial amount of migration \citep{1997Icar..126..261W}. Unfortunately, the strength and even direction of this effect is highly uncertain and depends on the detailed physical structure of the disk, along with the specifics of the thermodynamic model used \citep{2010MNRAS.408..876A, 2013A&A...549A.124B}. In addition, many STIPs do not lie in resonant chains, which should be a generic result of convergent migration \citep{2014MNRAS.445..749H}, although dynamical instabilities \citep{2007ApJ...654.1110T, 2019arXiv190709313M} or even turbulence \citep{2011A&A...531A...5P} can potentially disrupt resonant chains after they form.

%Alternatively, these planets may have formed in situ, being constructed only from the local mass budget of the disk. This may be the case for the gaseous envelopes that hot Jupiters attain near the inner edge of the disk \citep{2018ApJ...866L...2B}, although it is not clear if the cores of these worlds also formed locally. Additionally, magnetically driven disk winds, which are especially prevalent around M stars, have been shown to flatten the radial surface density profile, which suppresses type I migration \citep{2018A&A...615A..63O}.

%One of the first simulations of in situ short period terrestrial planet accretion was done by \citet{2007ApJ...669..606R}, who concluded that it was extremely difficult for this process to occur near the habitable zone of M stars. However, the initial conditions used were based off of the MMSN, scaled linearly by the mass of the host star. \citet{2013MNRAS.431.3444C} later showed that the rescaled MMSN tends to underpredict the mass budget of many extrasolar disks. Starting with a much higher surface density, \citet{2012ApJ...751..158H} were able to produce short period Earth-sized planets without invoking migration. A subsequent study by \citet{2014ApJ...795L..15S} argued that applying the MMSN analysis to some exoplanetary systems produced disks that were unphysical, assuming a gas to dust ratio of 1\%. There is no evidence, however that this ratio is universal and there is mounting evidence that shows that this quantity can vary radially \citep{2012ApJ...744..162A, 2014ApJ...788...59W}. It should also be noted that as of yet there are no direct measurements of the surface density profiles of the inner parts of protoplanetary disks. Additionally, the dust masses of disks measured with ALMA have been shown to sometimes underpredict the true values \citep{2019ApJ...878..116P, 2019ApJ...877L..18Z}.

%The question of whether or not in situ formation is viable for STIPs seems to rest largely on what the distribution of planetary embryos looked like that went on to assemble the final configuration of planets. In all of the aforementioned studies, the oligarchic growth model \citep{1998Icar..131..171K, 2000Icar..143...15K} is used to predict the sizes and spacings between the embryos going into the final phase of accretion. \citet{2002ApJ...581..666K} explored the influence of planetesimal orbital distance and surface density profile on the distribution of embryos, but did not extend the search of parameter space down to be of relevance to most STIPs.

%As we will present in a pilot study, runaway and oligarchic growth, which produces a bimodal population of embryos and dynamically hot, difficult to accrete planetesimals, does not appear to operate at short orbital periods. With a colder background population of planetesimals, growth should be more efficient and result in larger embryos. More massive embryos are also less affected by radial drift, which has been cited as a problem with short period in situ formation models \citep{2015MNRAS.448.1751I}. Given the lack of work that has been done on short period planetesimal accretion, along with the subsequent discovery that short period terrestrial worlds are incredibly common, a more careful examination of this phase of growth is warranted. This will place better context on the initial conditions used by recent simulations of short period late stage accretion.
